\RequirePackage{nag}
\documentclass[12pt,letterpaper]{article}
\usepackage{lipsum}

% \usepackage{fixltx2e}
% \usepackage{classicthesis}
\usepackage{polyglossia}
\usepackage[natbib=true, 
			style=numeric, %authoryear or numeric; comp == compact
			bibstyle=nature, 
			backend=biber,
			uniquelist=false, 
			sorting=none,
			uniquename=false]{biblatex}
\usepackage{booktabs}
\usepackage{relsize}
\usepackage{setspace}
\usepackage{lineno}
\usepackage{wrapfig}
\usepackage{sidecap}

\setdefaultlanguage{english}
\setmainfont[Mapping=tex-text, 
			 Numbers=OldStyle, 
			 % SizeFeatures={{Size=12}}
			 ]{Times}
\setsansfont[Mapping=tex-text, 
			 Numbers=OldStyle, 
			 % SizeFeatures={{Size=12}}
			 ]{Helvetica}
\setmonofont[Scale=0.8]{Monaco}
\setcounter{secnumdepth}{0}

\addbibresource{ELI.bib}

% Set the line spread (height). Be careful here, use too small rather than too
% large value. Also: double-spaced lines correspond to a value of ~1.3,
% depending on the font, NOT to 2.0
\setstretch{1.1} % 1.1 normally

\graphicspath{{figs/}}

%% Custom macros

\newcommand*\captitle[1]{\textbf{#1}}
\newcommand*\todo[1]{%
    \graffito{\textcolor{red}{TO\ DO: #1}}}

\newcommand*\gene[1]{\textit{#1}}
\newcommand*\ko[1]{\textit{#1\textsuperscript{\(-/-\)}}}
\newcommand*\protein[1]{#1}
\newcommand*\species[1]{\textit{#1}}


\title{\ruleline{Biographical Sketch}}
\lhead{Zhian N. Kamvar, Ph. D.}
\rhead{Biographical Sketch}
% The Biographical sketch must include the following information:

% * be no more than 2 pages in length, excluding lists of publications;
% * include a presentation of academic, extension, and/or research credentials
%   including, as applicable: earned degrees, teaching experience, employment
%   history, professional activities, honors and awards, and grants received;
% * indicate TOTAL number of undergraduate students directly mentored or trained
%   through experiential learning activities during the past four (4) years; and
% * include all relevant publications in refereed journals during the past four
%   (4) years, including those in submission.
% In addition to the information on Attach Biographical Sketch Field (see Part
% IV, C, 4 c), applicants must include the date of completion of their terminal
% degree. The Biographical sketch of the Primary mentor (more than one Primary
% Mentor if integrated project only) should include the number of mentees
% mentored at least for the last 5 years.

\begin{document}

\maketitle

\begin{center}
{\bf Zhian Namir Kamvar}\\
Department of Plant Pathology \\
University of Nebraska, Lincoln NE 68583\\
Phone: 541-286-0187; Email: \href{mailto:zkamvar@unl.edu}{zkamvar@unl.edu}
\end{center}

\subsection*{Education and Training}

\begin{center}
\begin{tabular}{llrr}%{p{0.25\linewidth}p{0.3\linewidth}p{0.15\linewidth}p{0.2\linewidth}}
\textbf{Institution} & \textbf{Area} & \textbf{Degree} & \textbf{Year Awarded}\\
Oregon State University & Botany and Plant Pathology & \textit{\textbf{Ph. D.}} & 2016-12-06\\
Truman State University & Biology  & \textit{\textbf{B. Sc.}} &  2007-12-17
\end{tabular}
\end{center}



\subsection*{Research and Professional Experience}

\begin{tabular}{p{0.15\linewidth}p{0.8\linewidth}}
2017 --	Present & Postdoctoral Research Assistant, Department of Plant Pathology 
			      University of Nebraska (UNL), Lincoln, NE\\
2011 -- 2016    & Graduate Research Assistant, Department of Botany and Plant 
                  Pathology, Oregon State University (OSU), Corvallis, OR
\end{tabular}

\subsection*{Current Research Focus}

My Ph.D. work focused on developing computational tools for the analysis of
clonal population genetic data, emphasizing the need for reproducibility in
genetic data analysis. I have carried this theme into my postdoctoral research,
focusing on determining if the host or region is important for driving genetic
diversity in the cosmopolitan plant pathogen, \textit{Sclerotinia sclerotiorum}.

\subsection*{Awards and Honors}

\begin{tabular}{p{0.125\linewidth}p{0.825\linewidth}}

2016 & Botany and Plant Pathology Grad. Student Assoc. Travel Award\\
2016 & APS Foundation Student Travel Award\\
2015 & APS Pacific Division Travel Award\\
2015 & OSU Graduate School Travel Award\\
2015 & National Evolutionary Synthesis Center (NESCent) Travel Award for Population Genetics in R Hackathon\\
2014 & OSU Botany and Plant Pathology Anita Summers Travel Award\\
2014 & Most Innovative [Radio] Program (\textit{Intercollegiate Broadcasting System})\\
2013 & Seattle Institute For Statistical Genetics Travel Award\\
2006 & Truman State University Summer Research Stipend\\
2003 & Truman State University Presidential Leadership Scholarship

\end{tabular}

\subsection*{Synergistic Activities}

\subsubsection*{Professional Service}
\begin{enumerate}
	
	\item \textit{\textbf{Ad-hoc peer review for:}} Molecular Ecology, Methods in Ecology and Evolution
	\item \textit{\textbf{Maintainer of:}} NESCent Population Genetics in R  website: 
                                 \href{http://popgen.nescent.org}{popgen.nescent.org}
\end{enumerate}
\subsubsection*{Outreach}

\begin{tabular}{p{0.15\linewidth}p{0.8\linewidth}}
2012 -- 2016 & Co-creator and Host, \textit{Inspiration Dissemination} --- 
			   award-winning science communication radio program interviewing 
			   graduate students about their lives and research at Oregon State 
			   University.\\
2011 -- 2015 & Volunteer at \textit{OSU Discovery Days} --- teaching K-8 students
               about the diversity of plants and fungi.
\end{tabular}

\subsubsection*{Teaching Experience}

\begin{tabular}{p{0.15\linewidth}p{0.8\linewidth}}
Summer 2017  & Introduction to R workshop, University of Nebraska-Lincoln and 
			   North-Central American Phytopathological Society Meeting,
			   Champaign, IL\\
Summer 2016  & Reproducible Research in R Workshop, American Phytopathological 
			   Society (APS) Annual Meeting, Tampa, FL\\
Spring 2016  & Botany 101: A Human Concern, Graduate Teaching Assistant, OSU\\
Summer 2015  & Population Genetics in R Workshop, APS Annual Meeting, Pasadena, CA\\
Summer 2014  & Population Genetics in R Workshop, APS Annual Meeting, Minneapolis, MN\\
Spring 2014  & Population Genetics in R Workshop, OSU\\
Winter 2012  & Biology 212: Intro to Biology, Graduate Teaching Assistant, OSU\\
2008 -- 2011 & English Instructor, Daegu, South Korea\\
Fall 2007    & Cell Biology, Undergraduate Teaching Assistant, TSU\\
Fall 2006    & Cell Biology, Undergraduate Teaching Assistant, TSU
\end{tabular}

\newpage

\subsubsection*{Peer-Reviewed Publications (last four years)}

\begin{enumerate}[leftmargin = 14pt]

	\item Gr\"unwald NJ, Everhart SE, Knaus BJ, \textbf{Kamvar ZN}. (2017)
	Best practices for population genetic analyses. \textit{Phytopathology}
	\textbf{In Press} doi: \href{http://dx.doi.org/10.1094/PHYTO-12-16-0425-RVW}{10.1094/PHYTO-12-16-0425-RVW}

	\vspace{6pt}

	\item Tabima JF, Everhart SE, Larsen MM, Weisberg AJ, \textbf{Kamvar ZN}, Tancos MA,
	Smart CD, Chang JH, Gr\"unwald NJ. (2016) Microbe-ID: an open source toolbox
	for microbial genotyping and species identification. \textit{PeerJ} \textbf{4}: e2279
	doi: \href{https://doi.org/10.7717/peerj.2279}{10.7717/peerj.2279}
	
	\vspace{6pt}
	
	\item Jombart T, Archer F, Schliep K, \textbf{Kamvar ZN}, Harris R, Paradis
	E, Goudet J, Lapp H (2016). apex: phylogenetics with multiple genes.
	\textit{Molecular Ecology Resources}. \textbf{17}:1 19-26 doi:
	\href{http://dx.doi.org/10.1111/1755-0998.12567}{10.1111/1755-0998.12567}
	
	\vspace{6pt}

	\item \textbf{Kamvar ZN}, L\'opez-Uribe MM, Coughlan S, Gr\"unwald NJ, Lapp
	H, Manel S (2016). Developing educational resources for population genetics
	in R: an open and collaborative approach. \textit{Molecular Ecology Resources}. 
	\textbf{17}:1 120-128 doi:
	\href{http://dx.doi.org/10.1111/1755-0998.12558}{10.1111/1755-0998.12558}
	
	\vspace{6pt}

	\item Gr\"unwald NJ, Larsen MM, \textbf{Kamvar ZN}, Reeser PW, Kanaskie A,
	Laine J, Wiese R (2015) First report of the EU1 clonal lineage of
	\textit{Phytophthora ramorum} on tanoak in an OR forest. 
	\textit{Plant Disease}. \textbf{100}:5, 1024-1024. doi:
	\href{http://dx.doi.org/10.1094/PDIS-10-15-1169-PDN}{10.1094/PDIS-10-15-1169-PDN}
	
	\vspace{6pt}

	\item \textbf{Kamvar ZN}, Brooks JC and Gr\"unwald NJ (2015) Novel R tools for
	analysis of genome-wide population genetic data with emphasis on clonality.
	\textit{Front. Genet.} \textbf{6}: 208. doi: \\
	\href{http://dx.doi.org/10.3389/fgene.2015.00208}{10.3389/fgene.2015.00208}
	
	\vspace{6pt}

	\item \textbf{Kamvar ZN}, Larsen MM, Kanaskie AM, Hansen EM, and Gr\"unwald
	NJ. (2015) Spatial and temporal analysis of populations of the sudden oak
	death pathogen in Oregon forests. \textit{Phytopathology}. \textbf{105}:7
	982-989. doi: 
	\href{http://dx.doi.org/10.1094/PHYTO-12-14-0350-FI}{10.1094/PHYTO-12-14-0350-FI}.
	
	\vspace{6pt}

	\item Weiland JE, Garrido PA, \textbf{Kamvar ZN}, Marek SM, Gr\"unwald NJ, and
	Garz\'on CD. (2015) Population structure of \textit{Pythium irregulare}, \textit{P.
	sylvaticum}, and \textit{P. ultimum} in forest nursery soils of Oregon and
	Washington. \textit{Phytopathology}. \textbf{105}:5 684-694. doi: \\
	\href{http://dx.doi.org/10.1094/PHYTO-05-14-0147-R}{10.1094/PHYTO-05-14-0147-R}
	
	\vspace{6pt}

    \item \textbf{Kamvar ZN}, Tabima JF, Gr\"unwald NJ. (2014) \textit{Poppr}: an
	R package for genetic analysis of populations with clonal, partially clonal,
	and/or sexual reproduction. PeerJ \textbf{2}: e281. doi: \\
	\href{http://dx.doi.org/10.7717/peerj.281}{10.7717/peerj.281}

\end{enumerate}

\end{document}
