\RequirePackage{nag}
\documentclass[12pt,letterpaper]{article}
\usepackage{atbegshi}
\usepackage{lipsum}

% \usepackage{fixltx2e}
% \usepackage{classicthesis}
\usepackage{polyglossia}
\usepackage[natbib=true, 
			style=numeric, %authoryear or numeric; comp == compact
			bibstyle=nature, 
			backend=biber,
			uniquelist=false, 
			sorting=none,
			uniquename=false]{biblatex}
\usepackage{booktabs}
\usepackage{relsize}
\usepackage{setspace}
\usepackage{lineno}
\usepackage{wrapfig}
\usepackage{sidecap}

\setdefaultlanguage{english}
\setmainfont[Mapping=tex-text, 
			 Numbers=OldStyle, 
			 % SizeFeatures={{Size=12}}
			 ]{Times}
\setsansfont[Mapping=tex-text, 
			 Numbers=OldStyle, 
			 % SizeFeatures={{Size=12}}
			 ]{Helvetica}
\setmonofont[Scale=0.8]{Monaco}
\setcounter{secnumdepth}{0}

\addbibresource{ELI.bib}

% Set the line spread (height). Be careful here, use too small rather than too
% large value. Also: double-spaced lines correspond to a value of ~1.3,
% depending on the font, NOT to 2.0
\setstretch{1.1} % 1.1 normally

\graphicspath{{figs/}}

%% Custom macros

\newcommand*\captitle[1]{\textbf{#1}}
\newcommand*\todo[1]{%
    \graffito{\textcolor{red}{TO\ DO: #1}}}

\newcommand*\gene[1]{\textit{#1}}
\newcommand*\ko[1]{\textit{#1\textsuperscript{\(-/-\)}}}
\newcommand*\protein[1]{#1}
\newcommand*\species[1]{\textit{#1}}


\title{\ruleline{Project Narrative}}


\begin{document}
\maketitle



\linenumbers

\section{Training/Career Development Plan}

The Training/Career Development Plan is a description of all activities that
the applicant plans to perform and participate to enhance the pre- or
postdoctoral training during the fellowship award period.

For Postdoctoral Fellowship applicants, a Training/Career Development Plan
includes plans for transition to career independence by development of
professional skills. These professional skills include teaching competencies;
what those career and training goals are; and results of the postdoctoral
fellow's previous and current research and scholarships that include
publications, presentations, etc.

\section{Mentoring Plan}

The applicants are expected to engage their mentors and/or advisors in the
development of their application. Thus, prior to submission of the
application, prospective fellows should already identify a Primary mentor who
will be willing to help them in their projects as well as professional
development (note: more than one Primary Mentor is acceptable for Integrated
Projects Only). If there are other collaborating mentors, their role and
responsibilities to the project and development of the applicant's skills
should be clearly described. For predoctoral applications, if the primary
mentor is not the student's graduate advisor or laboratory sponsor, the
relationship between advisor's work and the primary mentor's research should
be clearly defined, and the contribution of each individual in the student's
project as well as degree completion should be included. Because this is a
very important component of the project, the commitment of the mentor(s) is
included in the evaluation criteria as it pertains to project personnel. In
describing the role of the mentor, the applicant should:

\begin{enumerate}

  \item Briefly indicate how the mentoring and educational training will add to the
   skill sets of the National Institute of Food and Agriculture (NIFA) Fellow.

  \item Briefly explain the commitment of the primary mentor.

  \item Briefly describe the role of collaborating mentors (if applicable).

  \item With respect to the Primary Mentor, provide a list of former mentees and
   their current positions. 
   \begin{quote}
   NOTE: The Primary Mentor shall submit a Letter of Commitment (as an
   attachment to Field 12, Other Attachments, of the Other Project
   Information form-see section g. in the AFRI ELI RFA) explicitly
   indicating their respective responsibilities throughout the proposed
   project in relation to the Project Director.
   \end{quote}

  \item Briefly list and explain the role of other non-primary mentors.

\end{enumerate}


\section{Project Plan}

The research should be totally independent of the mentor's. Proven techniques
and technologies as part of the experimental approach, especially if these are
routinely employed, don't have to be provided in detail. Experimental
approaches or strategies including possible pitfalls and alternatives must be
provided in order to assess the overall feasibility of the Updated March 10,
2017 proposed study. Avoid open-ended screens or undefined outcomes. The scope
of the project should be within the 2-year timeframe.

\subsection{Introduction}

The introduction should include a well-defined problem, a clear statement of
the long-term goal(s), and supporting objectives of the proposed project.
Summarize the body of knowledge or other past activities that substantiate the
need for the proposed project. Describe ongoing or recently completed
activities related to the proposed project including the work of key project
personnel. Include preliminary data/information pertinent to the proposed
work. All works cited should be referenced (see Bibliography \& References
Cited, see section d. in the AFRI ELI RFA).


\subsection{Rationale and Significance}

\begin{itemize}
  \item Concisely present the rationale behind the proposed project and how it will
  advance the current knowledge in the field;

  \item Clearly describe the specific relationship of the project's objectives to
  one of the Program Area Priorities. The Program Area Priority(ies) must be
  specifically identified; and

  \item Describe how the proposed curricular activities (predoctoral) will support
  educational goals and project activities.
\end{itemize}

\subsection{Approach}

Provide a concise description of the proposed project and the problem(s) to be
addressed. Clearly describe the approaches to be used. Specifically, this
section must include:

\begin{itemize}
  \item A description of the project details proposed and the sequence in which the
   activities are to be performed;
 
  \item Methods to be used in carrying out the proposed project and feasibility of
   the methods (detail only if a new and unproven method is to be used; if
   employing commonly used methods provide information on the expertise
   available);
 
  \item Expected outcomes and outcome measures;
  \item Means by which results will be analyzed, assessed, or interpreted;
  \item How results or products will be used;
  \item Pitfalls that may be encountered, and possible alternatives;
  \item Limitations to proposed procedures;
 
  \item A full explanation of any materials, procedures, situations, or activities
   related to the project that may be hazardous to personnel, along with an
   outline or precautions to be exercised to avoid or mitigate the effects of
   such hazards;
 
  \item A timeline for attainment of objectives and for production of deliverables
   that includes annual milestones with specific, measurable outcomes; and
 
  \item Establishment of a profile on an established professional social networking
   site to document career progress during and beyond the duration of the
   Fellowship.

\end{itemize}




\section{Evaluation Plan}

A plan for evaluating progress towards objectives related to the
training/career development plan, mentoring plan, and project plan. The plan
must include milestones, which signify the completion of a major deliverable,
events, or accomplishment and serve to verify that the project is on schedule
and on track for successful conclusion. The plan should also include
descriptions of indicators that will be measured to evaluate whether the
education activities are successful in achieving project goals and contribute
to the achievement of the stated program goals and outcomes; and a
dissemination plan describing the methods that will be used to communicate
findings and project accomplishments.

In addition to the Project Narrative requirements above, the proposed
Integrated Project should clearly articulate:

\begin{itemize}

  \item Stakeholder involvement in project development, implementation, and
   evaluation, where appropriate;

  \item Objectives for each function included in the project (note that extension
   and education activities are expected to differ and to be described as
   separate project objectives; see enumerated descriptions in Part II, C
   (page 7 in the AFRI ELI RFA); and Updated March 10, 2017

  \item A dissemination plan describing the methods that will be used to
   communicate findings and project accomplishments.

  \item A plan for evaluating progress toward achieving project objectives must be
   included. The plan must include milestones, which signify the completion of
   a major deliverable, event, or accomplishment and serve to verify that the
   project is on schedule and on track for successful conclusion. The plan
   should also include descriptions of indicators that you will measure to
   evaluate whether the research, education, and/or extension activities are
   successful in achieving project goals and in contributing to achievement of
   the stated program goals and outcomes.

\end{itemize}


Test1 \citep{lehner2017independently} 
Test2 \citep{clarkson2017population}
Test3 \citep{clarkson2017population}.

% https://tex.stackexchange.com/a/224803/77699
\newpage
\AtBeginShipout{%
\AtBeginShipoutDiscard
}

\printbibliography

\end{document}
