\RequirePackage{nag}
\documentclass[12pt,letterpaper]{article}
\usepackage{lipsum}
% \usepackage{fixltx2e}
% \usepackage{classicthesis}
\usepackage{polyglossia}
\usepackage[natbib=true, 
			style=numeric, %authoryear or numeric; comp == compact
			bibstyle=nature, 
			backend=biber,
			uniquelist=false, 
			sorting=none,
			uniquename=false]{biblatex}
\usepackage{booktabs}
\usepackage{relsize}
\usepackage{setspace}
\usepackage{lineno}
\usepackage{wrapfig}
\usepackage{sidecap}

\setdefaultlanguage{english}
\setmainfont[Mapping=tex-text, 
			 Numbers=OldStyle, 
			 % SizeFeatures={{Size=12}}
			 ]{Times}
\setsansfont[Mapping=tex-text, 
			 Numbers=OldStyle, 
			 % SizeFeatures={{Size=12}}
			 ]{Helvetica}
\setmonofont[Scale=0.8]{Monaco}
\setcounter{secnumdepth}{0}

\addbibresource{ELI.bib}

% Set the line spread (height). Be careful here, use too small rather than too
% large value. Also: double-spaced lines correspond to a value of ~1.3,
% depending on the font, NOT to 2.0
\setstretch{1.1} % 1.1 normally

\graphicspath{{figs/}}

%% Custom macros

\newcommand*\captitle[1]{\textbf{#1}}
\newcommand*\todo[1]{%
    \graffito{\textcolor{red}{TO\ DO: #1}}}

\newcommand*\gene[1]{\textit{#1}}
\newcommand*\ko[1]{\textit{#1\textsuperscript{\(-/-\)}}}
\newcommand*\protein[1]{#1}
\newcommand*\species[1]{\textit{#1}}


\title{\ruleline{Project Summary}}
\lhead{Zhian N. Kamvar, Ph. D.}
\rhead{Project Summary}


% The summary should also include the relevance of the project to the goals of
% Agriculture and Food Research Initiative Food, Agriculture, Natural Resources
% and Human Sciences Education and Literacy Initiative (AFRI ELI) Competitive
% Grants Program. The following instructions are in addition to those included
% in section 4.7 of Part V of the NIFA Grants.gov Application Guide. Title the
% attachment as 'Project Summary' in the document header and save file as
% 'ProjectSummary'.

% The Project Summary must indicate the following:

% a. Names and institutions of the Project Director (PD) and Primary Mentor
%    (more than one mentor for Integrated project only);
% b. Predoctoral or Postdoctoral application
% c. Project type (research, education, extension, or integrated research,
%    education and/or extension)
% d. Indicate the primary AFRI Farm Bill Priority area focus of the project by
%    selecting one of the following six Farm Bill priority areas.
%     - Plant health and production and plant products;
%     - Animal health and production and animal products;
%     - Food safety, nutrition, and health;
%     - Bioenergy, natural resources, and environment;
%     - Agriculture systems and technology;
%     - Agriculture economics and rural communities
    
% The Project Summary should be a short, concise description of the research,
% education, extension or integrated research, education and/or extension
% project in the applicant's proposed doctoral program or postdoctoral training.
% The summary should also include the relevance of the project to the goals of
% AFRI ELI Fellowships Grant Program.


\begin{document}
\maketitle

\vspace{2em}

\begin{centering}
\begin{tabular}{ll}
\textbf{Project Director} & Zhian N. Kamvar (University of Nebraska-Lincoln)\\
\textbf{Primary Mentor  } & Sydney E. Everhart (University of Nebraska-Lincoln)\\
\textbf{Project Type    } & Integrated Teaching and Research\\
\textbf{Priority Area   } & Plant health and Production and Plant Products
\end{tabular}
\end{centering}

\vspace{2em}

\subsubsection*{Developing a Reproducible Research Curriculum From Real-World Examples in Agriculture}

Agricultural science research often directly impacts agricultural producers, and thus, it is imperative that production stakeholders are able to trust the viability and integrity of agricultural research. 
The practice of open and reproducible research, where all steps of the scientific process are made freely available and verifiable, holds many benefits for scientists including increased public confidence in science. 
The goals of this integrated post-doctoral project include: 1) creating a course on reproducible research using modern, active-learning techniques that utilize real-world examples in agriculture and (2) developing an example of a fully open and reproducible research project using a genomic investigation of thermal adaptation in the cosmopolitan plant pathogenic fungus \textit{Sclerotinia sclerotiorum} across the United States and China. 
\textit{S. sclerotiorum} is the causal agent of Sclerotinia Stem Rot and infects over 400 plant species, causing millions of dollars in damage in the United States, annually. 
Because understanding thermal adaptation will help management decisions in a changing climate and training in reproducible practices will have the long-term effect of improving the quality of scientific research, this project falls within the ``plant health and production and plant products'' foundational area. 
Under the mentorship of Dr. Sydney Everhart (University of Nebraska-Lincoln), this project will enhance my skills in science education and genomics research, preparing me well for a successful transition to a faculty position, and fulfilling the goal of the fellowship program to train the next generation of agricultural scientists.

\end{document}




