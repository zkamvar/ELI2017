\RequirePackage{nag}
\documentclass[12pt,letterpaper]{article}
\usepackage{lipsum}
% \usepackage{fixltx2e}
% \usepackage{classicthesis}
\usepackage{polyglossia}
\usepackage[natbib=true, 
			style=numeric,   % authoryear or numeric; comp == compact
			bibstyle=nature, 
			backend=biber,   % necessary for bibliography on separate page
			uniquelist=false, 
			sorting=none,
			uniquename=false]{biblatex}
\usepackage{booktabs}
\usepackage{fancyhdr}    % for header and footers
\usepackage{enumitem}    % to reduce vertical space in lists
\usepackage{relsize}
\usepackage{setspace}
\usepackage{notoccite}   % prevents citations in figures coming first
\usepackage{lineno}      % line numbers
\usepackage{wrapfig}     % allows for text to wrap figures (kinda clunky)
\usepackage{sidecap}
\usepackage{hyperref}    % hyperlinks/urls
\usepackage{indentfirst} % Indents first paragraph of section
\usepackage{titling}     % package to muck with the titles
\usepackage{titlesec}    % package to deal with the section titles
\usepackage[margin=1in]{geometry}

% Format spacing of the sections to be much smaller than normal
\titlespacing*
	{\section} % command
	{0em}      % left -1.25 is good, too
	{1.5ex}    % before separator
	{1ex}      % after separator

% Fancy Headers -------------------------------------------
\pagestyle{fancy}

% Reduce spacing in lists ---------------------------------
\setlist{nosep}

% Title Settings ------------------------------------------
\setlength{\droptitle}{-20ex} % raise the title
\pretitle{\begin{flushleft}\Large\bfseries\sffamily}
\posttitle{\end{flushleft}\vspace{-3ex}}
\preauthor{} % having a blank here means to not include it
\author{}
\postauthor{}
\predate{}
\date{}
\postdate{}

% Changing Title ------------------------------------------
% 
% Here, we center the title and place 5pt lines on either side.
% This is done by defining a new \hrulefill macro and adding a new
% command called \ruleline
%
% https://tex.stackexchange.com/a/65734/77699
\def\fhrulefill{\leavevmode\leaders\hrule height 5pt\hfill\kern0pt}
%
% https://tex.stackexchange.com/a/15122/77699
% Note: normally, this would be \hrulefill, but I wanted to change
% the thickness, so I made my own macro called \fhrulefill
\newcommand*\ruleline[1]{\par\noindent\raisebox{.3ex}{\makebox[\linewidth]{\fhrulefill\hspace{1ex}\raisebox{-.3ex}{#1}\hspace{1ex}\fhrulefill}}}


% Typeface Settings ----------------------------------------
\setdefaultlanguage{english}
\setmainfont[Mapping=tex-text, 
			 Numbers=OldStyle, 
			 SizeFeatures={{Size=12}}
			 ]{Times}
\setsansfont[Mapping=tex-text, 
			 Numbers=OldStyle, 
			 SizeFeatures={{Size=12}}
			 ]{Helvetica}
\setmonofont[Scale=0.8]{Monaco}
% \setcounter{secnumdepth}{0}


\addbibresource{ELI.bib}

% Set the line spread (height). Be careful here, use too small rather than too
% large value. Also: double-spaced lines correspond to a value of ~1.3,
% depending on the font, NOT to 2.0
\setstretch{1.0} % 1.1 normally

\graphicspath{{figs/}}

%% Custom macros

\newcommand*\captitle[1]{\textbf{#1}}
\newcommand*\todo[1]{%
    \graffito{\textcolor{red}{TO\ DO: #1}}}

\newcommand*\gene[1]{\textit{#1}}
\newcommand*\ko[1]{\textit{#1\textsuperscript{\(-/-\)}}}
\newcommand*\protein[1]{#1}
\newcommand*\species[1]{\textit{#1}}

\renewcommand\linenumberfont{\scriptsize\addfontfeatures{Numbers=Lining}}


\title{\ruleline{Project Summary}}
\lhead{Zhian N. Kamvar, Ph. D.}
\rhead{Project Summary}


% The summary should also include the relevance of the project to the goals of
% Agriculture and Food Research Initiative Food, Agriculture, Natural Resources
% and Human Sciences Education and Literacy Initiative (AFRI ELI) Competitive
% Grants Program. The following instructions are in addition to those included
% in section 4.7 of Part V of the NIFA Grants.gov Application Guide. Title the
% attachment as 'Project Summary' in the document header and save file as
% 'ProjectSummary'.

% The Project Summary must indicate the following:

% a. Names and institutions of the Project Director (PD) and Primary Mentor
%    (more than one mentor for Integrated project only);
% b. Predoctoral or Postdoctoral application
% c. Project type (research, education, extension, or integrated research,
%    education and/or extension)
% d. Indicate the primary AFRI Farm Bill Priority area focus of the project by
%    selecting one of the following six Farm Bill priority areas.
%     - Plant health and production and plant products;
%     - Animal health and production and animal products;
%     - Food safety, nutrition, and health;
%     - Bioenergy, natural resources, and environment;
%     - Agriculture systems and technology;
%     - Agriculture economics and rural communities
    
% The Project Summary should be a short, concise description of the research,
% education, extension or integrated research, education and/or extension
% project in the applicant's proposed doctoral program or postdoctoral training.
% The summary should also include the relevance of the project to the goals of
% AFRI ELI Fellowships Grant Program.


\begin{document}
\maketitle

\vspace{2em}

\begin{centering}
\begin{tabular}{ll}
\textbf{Project Director} & Zhian N. Kamvar (University of Nebraska-Lincoln)\\
\textbf{Primary Mentor  } & Sydney E. Everhart (University of Nebraska-Lincoln)\\
\textbf{Project Type    } & Integrated Teaching and Research\\
\textbf{Priority Area   } & Plant health and production and plant products
\end{tabular}
\end{centering}

\vspace{2em}

\subsubsection{Developing a Reproducible Research Curriculum From Real-World Examples in Agriculture}

The ``reproducibility crisis'' in science, in which results presented in publications cannot be reproduced with the same data and methods, has severely impacted public perception of scientific research in the United States.
Agricultural science research often has direct impacts on the producers, and it is imperative that these stakeholders hold trust in the results. 
The goals of this project is to (1) use modern active learning techniques and real-world examples to create a curriculum of reproducible research for students of agriculture and (2) Use a genomic investigation of local adaptation in the plant pathogenic fungus \textit{Sclerotinia sclerotiorum} as an example of fully open and reproducible research.
\textit{S. sclerotiorum} is a cosmopolitan necrotrophic fungus that infects >400 plant species and is the causal agent of Sclerotinia Stem Rot on canola, causing millions of dollars in damage in the US annually.
Genetic studies with ow resolution markers have shown that populations in China and the US are significantly differentiated, but these happened to be from sub-tropical and temperate climates, respectively. 
We hypothesize that these differences are driven by local adaptation to climate. 
We propose to sequence 96 isolates of \textit{S. sclerotiorum} across temperate and sub-tropical regions in the US and China. 
Tests for genomic signatures of local adaptation will be done using population genetic methods including the coalescent, phylogeographic methods, and variance partitioning.
By performing these analyses in an open and reproducible manner, we will (1) provide well-documented and curated raw genomic data for future analyses and (2) set an example of how large-scale genomic data can be analyzed in an open and reproducible manner.
Project results will include 96 whole genome sequences of \textit{S. sclerotiorum}, an insight into patterns of local adaptation of global populations of \textit{S. sclerotiorum}, and a reproducible research curriculum that will aid in improving research practices, which will increase stakeholder trust in the long term.


\end{document}
