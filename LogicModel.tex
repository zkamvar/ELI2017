\RequirePackage{nag}
\documentclass[12pt,letterpaper]{article}
\usepackage{lipsum}
% \usepackage{fixltx2e}
% \usepackage{classicthesis}
\usepackage{polyglossia}
\usepackage[natbib=true, 
			style=numeric, %authoryear or numeric; comp == compact
			bibstyle=nature, 
			backend=biber,
			uniquelist=false, 
			sorting=none,
			uniquename=false]{biblatex}
\usepackage{booktabs}
\usepackage{relsize}
\usepackage{setspace}
\usepackage{lineno}
\usepackage{wrapfig}
\usepackage{sidecap}

\setdefaultlanguage{english}
\setmainfont[Mapping=tex-text, 
			 Numbers=OldStyle, 
			 % SizeFeatures={{Size=12}}
			 ]{Times}
\setsansfont[Mapping=tex-text, 
			 Numbers=OldStyle, 
			 % SizeFeatures={{Size=12}}
			 ]{Helvetica}
\setmonofont[Scale=0.8]{Monaco}
\setcounter{secnumdepth}{0}

\addbibresource{ELI.bib}

% Set the line spread (height). Be careful here, use too small rather than too
% large value. Also: double-spaced lines correspond to a value of ~1.3,
% depending on the font, NOT to 2.0
\setstretch{1.1} % 1.1 normally

\graphicspath{{figs/}}

%% Custom macros

\newcommand*\captitle[1]{\textbf{#1}}
\newcommand*\todo[1]{%
    \graffito{\textcolor{red}{TO\ DO: #1}}}

\newcommand*\gene[1]{\textit{#1}}
\newcommand*\ko[1]{\textit{#1\textsuperscript{\(-/-\)}}}
\newcommand*\protein[1]{#1}
\newcommand*\species[1]{\textit{#1}}


\title{\ruleline{Logic Model}}
\lhead{Zhian N. Kamvar, Ph. D.}
\rhead{Logic Model}

\begin{document}
\maketitle


% \textbf{2 page limit}

% Include the elements of a logic model detailing the activities, outputs, and
% outcomes of the proposed project. The logic model planning process is a tool
% that should be used to develop your project before writing your application.
% This information may be provided as a narrative or formatted into a logic model
% chart. More information and resources related to the logic model planning
% process are provided at \\
% \url{https://nifa.usda.gov/resource/integrated-programs-logic-model-planning-process}.

\section{Situation}

% A description of the challenge or opportunity. The problem or issue to be
% addressed, within a complex of socio-political, environmental, and economic
% conditions.

Currently, there is no standard for data management, open science, or reproducible analyses in plant pathology; community supported education materials are underused; and there is no mechanism to receive credit for open data and materials.

% Education in agricultural data analysis needs a revitalization.
% Current educational materials enforce best practices in experimental design and statistical theory, but fall short concerning data management, open science, and reproducible research.
% Encouraging best practices in these areas have widespread benefits for the agricultural science community.
% Open and well-managed data not only means that the results are easily verifiable, but that they can easily be built upon, and the data have a life outside of the original experiment.
% These can have the effect of increased stakeholder trust, more opportunities for collaboration and funding, and a wealth of material for educators.

% Many researchers are wary of sharing data, however.
% Some fear they would not receive credit if these data are used by other researchers.
% Others may not want to spend the time or energy to curate the data so that it is easily accessible. 
% It is clear that there do not currently exist adequate incentives to data sharing. 

% A solution to improving these practices in agricultural sciences may be to take an integrated approach where scientists share workflows associated with publications as cite-able education material. 
% The American Phytopathological Society currently has an open on-line Education
% Center (\url{http://www.apsnet.org/edcenter/Pages/default.aspx}), which hosts
% peer-reviewed materials, but it has not seen any new material since 2014.
% This resource could be used to host open data and dynamic workflows generated by plant pathologists, which would allow data to be open for educational use and reuse, providing all of the benefits of open science while removing many of the perceived drawbacks.


\section{Inputs}

% What is invested, such as resources, contributions, and investments that are
% provided for the program.

\begin{itemize}
	\item postdoctoral researcher
	\item leaders in different areas of the plant pathology community
	\item APS Education Center (\url{http://www.apsnet.org/edcenter/Pages/default.aspx})
	\item APS Annual and Regional Meetings
\end{itemize}

\section{Activities}

% what the program does with its inputs to services it provides to fulfill its
% mission.

\begin{itemize}
	\item create open and accessible resource for sharing open data and materials
	\item incorporate resource with APS Education Center
	\item prepare supplementary educational workshops on open science to engage leaders in the community 
	\item conduct novel research on population genomics of white-mold pathogen \textit{Sclerotinia sclerotiorum}
	\item organize bootcamps at national and regional APS meetings to prepare workflows for submission to resource
\end{itemize}

\section{Outputs}

% Products, services and events that are intended to lead to the program's
% outcomes.
\begin{itemize}
	\item A resource for depositing open workflows as educational materials in plant pathology
	\item Open and cite-able data for re-use
	\item An active community dedicated to sharing data
\end{itemize}

\section{Outcomes}

% Planned results or changes for individuals, groups, communities, organizations
% or systems.

\subsection{Change in knowledge}

% Occurs when there is a change in knowledge or the participants actually learn.
Open data means that these can be used to provide engaging and relevant examples for plant pathology education. It also means that these data can be used to improve models and theory. 

\subsection{Change in behavior}

% Occurs when there is a change in behavior or the participants act upon what
% they have learned.
Researchers within the community will be more open with their data and contributions to the Education Center will increase.

\subsection{Change in condition}

% Occurs when a societal condition is improved.
\begin{itemize}
	\item increased stakeholder trust
	\item open materials increases accessibility 
	\item data has a life beyond one study
	\item a more open and collaborative environment for newcomers
\end{itemize}

\section{External factors}

% Variables that may have an effect on the portfolio, program, or project but
% which cannot be changed by the managers of the portfolio, program, or project.

Much of this depends on the willingness of the plant pathology community and APS Education Center to adopt open science practices. 

\section{Assumptions}

% The premises based on theory, research, evaluation knowledge, etc. that
% support the relationships of the elements of the logic model and upon which
% the success of the portfolio, program, or project rests.

Open science practices can have widespread benefits to the agricultural sciences community including easily verifiable results, stakeholder trust, and a plethora of resources for the development of educational materials. 

There is currently a growing interest in reproducible research within plant pathology.


\end{document}
