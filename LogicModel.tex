\RequirePackage{nag}
\documentclass[12pt,letterpaper]{article}
\usepackage{lipsum}
% \usepackage{fixltx2e}
% \usepackage{classicthesis}
\usepackage{polyglossia}
\usepackage[natbib=true, 
			style=numeric, %authoryear or numeric; comp == compact
			bibstyle=nature, 
			backend=biber,
			uniquelist=false, 
			sorting=none,
			uniquename=false]{biblatex}
\usepackage{booktabs}
\usepackage{relsize}
\usepackage{setspace}
\usepackage{lineno}
\usepackage{wrapfig}
\usepackage{sidecap}

\setdefaultlanguage{english}
\setmainfont[Mapping=tex-text, 
			 Numbers=OldStyle, 
			 % SizeFeatures={{Size=12}}
			 ]{Times}
\setsansfont[Mapping=tex-text, 
			 Numbers=OldStyle, 
			 % SizeFeatures={{Size=12}}
			 ]{Helvetica}
\setmonofont[Scale=0.8]{Monaco}
\setcounter{secnumdepth}{0}

\addbibresource{ELI.bib}

% Set the line spread (height). Be careful here, use too small rather than too
% large value. Also: double-spaced lines correspond to a value of ~1.3,
% depending on the font, NOT to 2.0
\setstretch{1.1} % 1.1 normally

\graphicspath{{figs/}}

%% Custom macros

\newcommand*\captitle[1]{\textbf{#1}}
\newcommand*\todo[1]{%
    \graffito{\textcolor{red}{TO\ DO: #1}}}

\newcommand*\gene[1]{\textit{#1}}
\newcommand*\ko[1]{\textit{#1\textsuperscript{\(-/-\)}}}
\newcommand*\protein[1]{#1}
\newcommand*\species[1]{\textit{#1}}


\title{\ruleline{Logic Model}}

\begin{document}
\maketitle
\linenumbers

\textbf{2 page limit}

Include the elements of a logic model detailing the activities, outputs, and
outcomes of the proposed project. The logic model planning process is a tool
that should be used to develop your project before writing your application.
This information may be provided as a narrative or formatted into a logic model
chart. More information and resources related to the logic model planning
process are provided at
\url{https://nifa.usda.gov/resource/integrated-programs-logic-model-planning-process}.

\hline

\section{Situation}

% A description of the challenge or opportunity. The problem or issue to be
% addressed, within a complex of socio-political, environmental, and economic
% conditions.

Education in plant pathology needs a revitalization. While excellent, peer-
reviewed materials are available at the American Phytopathological Society (APS)
\href{http://www.apsnet.org/edcenter/Pages/default.aspx}{Education Center}, 
these do not adequately provide current and future plant pathologists with the
tools necessary to analyze data openly or reproducibly. Lessons on statistical
methods often describe where the data came from, but leave the student mystified
as to how the data are stored, cleaned, validated, and curated.

In recent years, plant pathologists have shown a keen interest in open and
reproducible analysis, as demonstrated by sold-out workshops on data analysis in
R at APS National and Regional meetings. Moreover, in the APS Education Center,
there are currently no lessons covering design and analysis of whole genome
sequencing (wgs) experiments that can range from systematic questions to
metagenomic analyses.

What's needed now in plant pathology are educational resources for open and
reproducible research. We need researchers to open up their data and analyses
and to lead by example.

\section{Inputs}

% What is invested, such as resources, contributions, and investments that are
% provided for the program.


\section{Activities}

% what the program does with its inputs to services it provides to fulfill its
% mission.

\section{Outputs}

% Products, services and events that are intended to lead to the program's
% outcomes.

\section{Outcomes}

% Planned results or changes for individuals, groups, communities, organizations
% or systems.

\subsection{Change in knowledge}

% Occurs when there is a change in knowledge or the participants actually learn.

\subsection{Change in behavior}

% Occurs when there is a change in behavior or the participants act upon what
% they have learned.

\subsection{Change in condition}

% Occurs when a societal condition is improved.

\section{External factors}

% Variables that may have an effect on the portfolio, program, or project but
% which cannot be changed by the managers of the portfolio, program, or project.

\section{Assumptions}

% The premises based on theory, research, evaluation knowledge, etc. that
% support the relationships of the elements of the logic model and upon which
% the success of the portfolio, program, or project rests.


\end{document}
