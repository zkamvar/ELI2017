\RequirePackage{nag}
\documentclass[12pt,letterpaper]{article}
\usepackage{lipsum}
% \usepackage{fixltx2e}
% \usepackage{classicthesis}
\usepackage{polyglossia}
\usepackage[natbib=true, 
			style=numeric, %authoryear or numeric; comp == compact
			bibstyle=nature, 
			backend=biber,
			uniquelist=false, 
			sorting=none,
			uniquename=false]{biblatex}
\usepackage{booktabs}
\usepackage{relsize}
\usepackage{setspace}
\usepackage{lineno}
\usepackage{wrapfig}
\usepackage{sidecap}

\setdefaultlanguage{english}
\setmainfont[Mapping=tex-text, 
			 Numbers=OldStyle, 
			 % SizeFeatures={{Size=12}}
			 ]{Times}
\setsansfont[Mapping=tex-text, 
			 Numbers=OldStyle, 
			 % SizeFeatures={{Size=12}}
			 ]{Helvetica}
\setmonofont[Scale=0.8]{Monaco}
\setcounter{secnumdepth}{0}

\addbibresource{ELI.bib}

% Set the line spread (height). Be careful here, use too small rather than too
% large value. Also: double-spaced lines correspond to a value of ~1.3,
% depending on the font, NOT to 2.0
\setstretch{1.1} % 1.1 normally

\graphicspath{{figs/}}

%% Custom macros

\newcommand*\captitle[1]{\textbf{#1}}
\newcommand*\todo[1]{%
    \graffito{\textcolor{red}{TO\ DO: #1}}}

\newcommand*\gene[1]{\textit{#1}}
\newcommand*\ko[1]{\textit{#1\textsuperscript{\(-/-\)}}}
\newcommand*\protein[1]{#1}
\newcommand*\species[1]{\textit{#1}}


\title{\ruleline{Data Management Plan}}
\lhead{Zhian N. Kamvar, Ph. D.}
\rhead{Data Management Plan}

\begin{document}
\maketitle

\section{Expected Data Type}

Primary data from research activities will be digital DNA sequence reads generated from whole genome sequencing. Primary data from educational activities will consist of curriculum and the number of students attending the course, stored digitally. 

\section{Data Format}

The primary data format for research activities will be gzip compressed FASTQ files containing both the sequence and quality scores. Secondary data formats include Makefiles, BASH scripts, Python scripts, and R scripts used for analysis in addition to text files describing the data processing.
These will be accompanied by virtual machines (Docker containers) that will preserve identical computing environments for reproducibility. 

The primary data format for educational activities will be markdown-formatted text files and rendered HTML files along with all the scripts used for website generation. 

\section{Data Storage and Preservation}

% As stated in the proposal instructions, "outline strategies, tools, and
% contingency plans that will be used to avoid data loss, degradation, or
% damage". It may be beneficial to briefly describe these contingency plans.

All data will be stored free of charge in the Open Science Framework (\url{https://osf.io}) and additionally archived on the CERN-funded Zenodo (\url{https://zenodo.org/}). 
Both organizations store their data distributed across data centers with daily backups and quality checks to prevent data loss, degradation, or damage.
Zenodo uses CERN data centers in Geneva, mirrored in Budapest, with retention guaranteed for the next 20 years. 
In the event of closure, Zenodo has outlined plans to transfer data to appropriate external repositories.
THe Open Science Framework uses Rackspace and Amazon Glacier for data storage and have an established preservation fund of \$250,000 to fund data storage for 50+ years in the case of termination of the Open Science Framework. 
In addition, sequencing reads and assembled genomes will be uploaded to the NCBI Short Read Archive and GenBank. 
As both Zenodo and the Open Science Framework are free of charge, no funds will be required for data storage and preservation.

\section{Data Sharing and Public Access}

All data and materials will be hosted or mirrored on the Open Science Framework.

Data from the research component will be available under the Open Data Commons (ODC) Open Database License 1.0 and code will be available under the Massachusetts Institute of Technology (MIT) License. 

Data from the teaching component will be released to the Public Domain, hosted on GitHub (https://github.com), and archived on Zenodo.



\section{Roles and Responsibilities}

Zhian Namir Kamvar will be responsible for all aspects of data management, and Sydney E. Everhart will verify all appropriate actions are taken. 

% \textbf{2 page limit}

% The DMP should clearly articulate how the project director (PD) and co-PDs plan
% to manage and disseminate data generated by the project. NIFA and reviewers will
% consider the DMP during the merit review process. NIFA is aware of the need to
% provide flexibility in assessing DMPs. The DMP should contain the following
% components.

% \begin{itemize}

%   \item \textbf{Expected Data Type} Describe the type of data (e.g., digital,
% 	non- digital), how they will be generated, and whether the data are primary
% 	or metadata. Examples include: for research, lab work, field work and
% 	surveys; for education, number of students enrolled/participated, degrees
% 	granted, curriculum, and training products; for extension, outreach
% 	materials, number of stakeholders reached, number of activities, and
% 	assessment questionnaires.
%   \item \textbf{Data Format}
% 	For data to be readily accessible and usable, it is critical to use an
% 	appropriate community-recognized standard and machine readable formats when
% 	they exist. The data should preferentially be stored in recognized public
% 	databases. Regardless of the format used that data set should contain enough
% 	information to allow independent use of the data.
%   \item \textbf{Data Storage and Preservation}
% 	Data should be stored in a safe environment with adequate measures taken for
% 	its long-term preservation. Applicants should describe plans for storing and
% 	preserving their data during and after the project and specify the data
% 	repositories, if they exist. They should outline strategies, tools, and
% 	contingency plans that will be used to avoid data loss, degradation, or
% 	damage.
%   \item \textbf{Data Sharing and Public Access}
% 	Describe your data access and sharing procedures during and after the grant.
% 	Provide any restrictions such as copyright, confidentiality, patent,
% 	appropriate credit, disclaimers, or conditions for use of the data by other
% 	parties.
%   \item \textbf{Roles and Responsibilities}
% 	Who will ensure DMP implementation? This is particularly important for
% 	multi- investigator and multi-institutional projects. Provide a contingency
% 	plan in case key personnel leave the project. Also, what resources will be
% 	needed for the DMP? If funds are needed, have they been added to the budget
% 	request and budget narrative? Projects must budget sufficient resources to
% 	develop and implement the proposed DMP.

% \end{itemize}

\end{document}
