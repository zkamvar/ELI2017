\RequirePackage{nag}
\documentclass[12pt,letterpaper]{article}
\usepackage{lipsum}
% \usepackage{fixltx2e}
% \usepackage{classicthesis}
\usepackage{polyglossia}
\usepackage[natbib=true, 
			style=numeric,   % authoryear or numeric; comp == compact
			bibstyle=nature, 
			backend=biber,   % necessary for bibliography on separate page
			uniquelist=false, 
			sorting=none,
			uniquename=false]{biblatex}
\usepackage{booktabs}
\usepackage{fancyhdr}    % for header and footers
\usepackage{enumitem}    % to reduce vertical space in lists
\usepackage{relsize}
\usepackage{setspace}
\usepackage{notoccite}   % prevents citations in figures coming first
\usepackage{lineno}      % line numbers
\usepackage{wrapfig}     % allows for text to wrap figures (kinda clunky)
\usepackage{sidecap}
\usepackage{hyperref}    % hyperlinks/urls
\usepackage{indentfirst} % Indents first paragraph of section
\usepackage{titling}     % package to muck with the titles
\usepackage{titlesec}    % package to deal with the section titles
\usepackage[margin=1in]{geometry}

% Format spacing of the sections to be much smaller than normal
\titlespacing*
	{\section} % command
	{0em}      % left -1.25 is good, too
	{1.5ex}    % before separator
	{1ex}      % after separator

% Fancy Headers -------------------------------------------
\pagestyle{fancy}

% Reduce spacing in lists ---------------------------------
\setlist{nosep}

% Title Settings ------------------------------------------
\setlength{\droptitle}{-20ex} % raise the title
\pretitle{\begin{flushleft}\Large\bfseries\sffamily}
\posttitle{\end{flushleft}\vspace{-3ex}}
\preauthor{} % having a blank here means to not include it
\author{}
\postauthor{}
\predate{}
\date{}
\postdate{}

% Changing Title ------------------------------------------
% 
% Here, we center the title and place 5pt lines on either side.
% This is done by defining a new \hrulefill macro and adding a new
% command called \ruleline
%
% https://tex.stackexchange.com/a/65734/77699
\def\fhrulefill{\leavevmode\leaders\hrule height 5pt\hfill\kern0pt}
%
% https://tex.stackexchange.com/a/15122/77699
% Note: normally, this would be \hrulefill, but I wanted to change
% the thickness, so I made my own macro called \fhrulefill
\newcommand*\ruleline[1]{\par\noindent\raisebox{.3ex}{\makebox[\linewidth]{\fhrulefill\hspace{1ex}\raisebox{-.3ex}{#1}\hspace{1ex}\fhrulefill}}}


% Typeface Settings ----------------------------------------
\setdefaultlanguage{english}
\setmainfont[Mapping=tex-text, 
			 Numbers=OldStyle, 
			 SizeFeatures={{Size=12}}
			 ]{Times}
\setsansfont[Mapping=tex-text, 
			 Numbers=OldStyle, 
			 SizeFeatures={{Size=12}}
			 ]{Helvetica}
\setmonofont[Scale=0.8]{Monaco}
% \setcounter{secnumdepth}{0}


\addbibresource{ELI.bib}

% Set the line spread (height). Be careful here, use too small rather than too
% large value. Also: double-spaced lines correspond to a value of ~1.3,
% depending on the font, NOT to 2.0
\setstretch{1.0} % 1.1 normally

\graphicspath{{figs/}}

%% Custom macros

\newcommand*\captitle[1]{\textbf{#1}}
\newcommand*\todo[1]{%
    \graffito{\textcolor{red}{TO\ DO: #1}}}

\newcommand*\gene[1]{\textit{#1}}
\newcommand*\ko[1]{\textit{#1\textsuperscript{\(-/-\)}}}
\newcommand*\protein[1]{#1}
\newcommand*\species[1]{\textit{#1}}

\renewcommand\linenumberfont{\scriptsize\addfontfeatures{Numbers=Lining}}


\title{\ruleline{Data Management Plan}}
\lhead{Zhian N. Kamvar, Ph. D.}
\rhead{Data Management Plan}

\begin{document}
\maketitle

\subsection{Expected Data Type}

Primary data from research activities are expected to be digital DNA sequence reads generated from whole genome sequencing. Primary data from educational activities consist of curriculum and the number of students attending the course, stored digitally. 

\subsection{Data Format}

The primary data format for research activities will be gzip compressed FASTQ files containing both the sequence and quality scores. Secondary data formats include Makefiles, BASH scripts, Python scripts, and R scripts used for analysis in addition to text files describing the data processing.
These will be accompanied by virtual machines (Docker containers) that will preserve identical computing environments for reproducibility. 

The primary data format for educational activities will be in the form of markdown-formatted text files and rendered HTML files along with all the scripts used for website generation. 

\subsection{Data Storage and Preservation}

All data will be stored free of charge in the Open Science Framework (\url{https://osf.io}) and additionally archived on the CERN-funded Zenodo (\url{https://zenodo.org/}). Both organizations have contingency plants that will allow data preservation for 25+ years in the case that primary funding runs out. In addition, sequencing reads and assembled genomes will be uploaded to the NCBI Short Read Archive and GenBank. 

\subsection{Data Sharing and Public Access}

Data from the research component will be available under the ODC Open Database License 1.0 and code will be available under the MIT license. 

Data from the teaching component will be released to the Public Domain.

All data and materials will be hosted at the Open Science Framework.

\subsection{Roles and Responsibilities}

Zhian Namir Kamvar will be responsible for all aspects of data management, and Sydney E. Everhart will verify all appropriate actions are taken. 

% \textbf{2 page limit}

% The DMP should clearly articulate how the project director (PD) and co-PDs plan
% to manage and disseminate data generated by the project. NIFA and reviewers will
% consider the DMP during the merit review process. NIFA is aware of the need to
% provide flexibility in assessing DMPs. The DMP should contain the following
% components.

% \begin{itemize}

%   \item \textbf{Expected Data Type} Describe the type of data (e.g., digital,
% 	non- digital), how they will be generated, and whether the data are primary
% 	or metadata. Examples include: for research, lab work, field work and
% 	surveys; for education, number of students enrolled/participated, degrees
% 	granted, curriculum, and training products; for extension, outreach
% 	materials, number of stakeholders reached, number of activities, and
% 	assessment questionnaires.
%   \item \textbf{Data Format}
% 	For data to be readily accessible and usable, it is critical to use an
% 	appropriate community-recognized standard and machine readable formats when
% 	they exist. The data should preferentially be stored in recognized public
% 	databases. Regardless of the format used that data set should contain enough
% 	information to allow independent use of the data.
%   \item \textbf{Data Storage and Preservation}
% 	Data should be stored in a safe environment with adequate measures taken for
% 	its long-term preservation. Applicants should describe plans for storing and
% 	preserving their data during and after the project and specify the data
% 	repositories, if they exist. They should outline strategies, tools, and
% 	contingency plans that will be used to avoid data loss, degradation, or
% 	damage.
%   \item \textbf{Data Sharing and Public Access}
% 	Describe your data access and sharing procedures during and after the grant.
% 	Provide any restrictions such as copyright, confidentiality, patent,
% 	appropriate credit, disclaimers, or conditions for use of the data by other
% 	parties.
%   \item \textbf{Roles and Responsibilities}
% 	Who will ensure DMP implementation? This is particularly important for
% 	multi- investigator and multi-institutional projects. Provide a contingency
% 	plan in case key personnel leave the project. Also, what resources will be
% 	needed for the DMP? If funds are needed, have they been added to the budget
% 	request and budget narrative? Projects must budget sufficient resources to
% 	develop and implement the proposed DMP.

% \end{itemize}

\end{document}
