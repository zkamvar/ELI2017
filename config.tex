% \usepackage{fixltx2e}
% \usepackage{classicthesis}
\usepackage{polyglossia}
\usepackage[natbib=true, 
			style=numeric,   % authoryear or numeric; comp == compact
			bibstyle=nature, 
			backend=biber,   % necessary for bibliography on separate page
			uniquelist=false, 
			sorting=none,
			uniquename=false]{biblatex}
\usepackage{booktabs}
\usepackage{fancyhdr}    % for header and footers
\usepackage{enumitem}    % to reduce vertical space in lists
\usepackage{relsize}
\usepackage{setspace}
\usepackage{lineno}      % line numbers
\usepackage{wrapfig}     % allows for text to wrap figures (kinda clunky)
\usepackage{sidecap}
\usepackage{hyperref}    % hyperlinks/urls
\usepackage{indentfirst} % Indents first paragraph of section
\usepackage{titling}     % package to muck with the titles
\usepackage{titlesec}    % package to deal with the section titles
\usepackage[margin=1in]{geometry}

% Format spacing of the sections to be much smaller than normal
\titlespacing*
	{\section} % command
	{0em}      % left -1.25 is good, too
	{1.5ex}    % before separator
	{1ex}      % after separator

% Fancy Headers -------------------------------------------
\pagestyle{fancy}

% Reduce spacing in lists ---------------------------------
\setlist{noitemsep}

% Title Settings ------------------------------------------
\setlength{\droptitle}{-20ex} % raise the title
\pretitle{\begin{flushleft}\Large\bfseries\sffamily}
\posttitle{\end{flushleft}\vspace{-3ex}}
\preauthor{} % having a blank here means to not include it
\author{}
\postauthor{}
\predate{}
\date{}
\postdate{}

% Changing Title ------------------------------------------
% 
% Here, we center the title and place 5pt lines on either side.
% This is done by defining a new \hrulefill macro and adding a new
% command called \ruleline
%
% https://tex.stackexchange.com/a/65734/77699
\def\fhrulefill{\leavevmode\leaders\hrule height 5pt\hfill\kern0pt}
%
% https://tex.stackexchange.com/a/15122/77699
% Note: normally, this would be \hrulefill, but I wanted to change
% the thickness, so I made my own macro called \fhrulefill
\newcommand*\ruleline[1]{\par\noindent\raisebox{.3ex}{\makebox[\linewidth]{\fhrulefill\hspace{1ex}\raisebox{-.3ex}{#1}\hspace{1ex}\fhrulefill}}}


% Typeface Settings ----------------------------------------
\setdefaultlanguage{english}
\setmainfont[Mapping=tex-text, 
			 Numbers=OldStyle, 
			 SizeFeatures={{Size=12}}
			 ]{Times}
\setsansfont[Mapping=tex-text, 
			 Numbers=OldStyle, 
			 SizeFeatures={{Size=12}}
			 ]{Helvetica}
\setmonofont[Scale=0.8]{Monaco}
\setcounter{secnumdepth}{0}


\addbibresource{ELI.bib}

% Set the line spread (height). Be careful here, use too small rather than too
% large value. Also: double-spaced lines correspond to a value of ~1.3,
% depending on the font, NOT to 2.0
\setstretch{1.1} % 1.1 normally

\graphicspath{{figs/}}

%% Custom macros

\newcommand*\captitle[1]{\textbf{#1}}
\newcommand*\todo[1]{%
    \graffito{\textcolor{red}{TO\ DO: #1}}}

\newcommand*\gene[1]{\textit{#1}}
\newcommand*\ko[1]{\textit{#1\textsuperscript{\(-/-\)}}}
\newcommand*\protein[1]{#1}
\newcommand*\species[1]{\textit{#1}}

\renewcommand\linenumberfont{\scriptsize\addfontfeatures{Numbers=Lining}}
